\nonstopmode{}
\documentclass[a4paper]{book}
\usepackage[times,inconsolata,hyper]{Rd}
\usepackage{makeidx}
\usepackage[utf8]{inputenc} % @SET ENCODING@
% \usepackage{graphicx} % @USE GRAPHICX@
\makeindex{}
\begin{document}
\chapter*{}
\begin{center}
{\textbf{\huge Package `dockerfiler'}}
\par\bigskip{\large \today}
\end{center}
\begin{description}
\raggedright{}
\inputencoding{utf8}
\item[Title]\AsIs{Easy Dockerfile Creation from R}
\item[Version]\AsIs{0.1.3}
\item[Description]\AsIs{Build a Dockerfile straight from your R session. 'dockerfiler' allows
you to create step by step a Dockerfile, and provide convenient tools to wrap
R code inside this Dockerfile.}
\item[License]\AsIs{MIT + file LICENSE}
\item[URL]\AsIs{}\url{https://github.com/ColinFay/dockerfiler}\AsIs{}
\item[BugReports]\AsIs{}\url{https://github.com/ColinFay/dockerfiler/issues}\AsIs{}
\item[Imports]\AsIs{attempt, glue, R6, utils}
\item[Suggests]\AsIs{covr, testthat, knitr, rmarkdown}
\item[Encoding]\AsIs{UTF-8}
\item[LazyData]\AsIs{true}
\item[RoxygenNote]\AsIs{6.1.0}
\item[VignetteBuilder]\AsIs{knitr}
\item[NeedsCompilation]\AsIs{no}
\item[Author]\AsIs{Colin Fay [cre, aut]}
\item[Maintainer]\AsIs{Colin Fay }\email{contact@colinfay.me}\AsIs{}
\item[Repository]\AsIs{CRAN}
\item[Date/Publication]\AsIs{2019-03-19 16:10:03 UTC}
\end{description}
\Rdcontents{\R{} topics documented:}
\inputencoding{utf8}
\HeaderA{Dockerfile}{A Dockerfile template}{Dockerfile}
\keyword{datasets}{Dockerfile}
%
\begin{Description}\relax
A Dockerfile template
\end{Description}
%
\begin{Usage}
\begin{verbatim}
Dockerfile
\end{verbatim}
\end{Usage}
%
\begin{Format}
An object of class \code{R6ClassGenerator} of length 24.
\end{Format}
%
\begin{Value}
A dockerfile template
\end{Value}
%
\begin{Section}{Methods}

\begin{description}

\item[\code{RUN}] add a RUN command
\item[\code{ADD}] add a ADD command
\item[\code{COPY}] add a COPY command
\item[\code{WORKDIR}] add a WORKDIR command
\item[\code{EXPOSE}] add an EXPOSE command
\item[\code{VOLUME}] add a VOLUME command
\item[\code{CMD}] add a CMD command
\item[\code{LABEL}] add a LABEL command
\item[\code{ENV}] add a ENV command
\item[\code{ENTRYPOINT}] add a ENTRYPOINT command
\item[\code{VOLUME}] add a VOLUME command
\item[\code{USER}] add a USER command
\item[\code{ARG}] add an ARG command
\item[\code{ONBUILD}] add a ONBUILD command
\item[\code{STOPSIGNAL}] add a STOPSIGNAL command
\item[\code{HEALTHCHECK}] add a HEALTHCHECK command
\item[\code{STOPSIGNAL}] add a STOPSIGNAL command
\item[\code{SHELL}] add a SHELL command
\item[\code{MAINTAINER}] add a MAINTAINER command
\item[\code{custom}] add a custom command
\item[\code{write}] save the Dockerfile
\item[\code{switch\_cmd}] switch two command
\item[\code{remove\_cmd}] remove\_cmd one or more command(s)

\end{description}

\end{Section}
%
\begin{Examples}
\begin{ExampleCode}
my_dock <- Dockerfile$new()
\end{ExampleCode}
\end{Examples}
\inputencoding{utf8}
\HeaderA{dock\_from\_desc}{Docker file from DESCRIPTION}{dock.Rul.from.Rul.desc}
%
\begin{Description}\relax
This function will parse a DESCRIPTION file and
create a Dockerfile that remotes::install\_cran
all the functions from the Imports section, add
the COPY of the tar.gz of the package, and install
the package.
\end{Description}
%
\begin{Usage}
\begin{verbatim}
dock_from_desc(path = "DESCRIPTION", FROM = "rocker/r-base",
  AS = NULL)
\end{verbatim}
\end{Usage}
%
\begin{Arguments}
\begin{ldescription}
\item[\code{path}] Path to DESCRIPTION

\item[\code{FROM}] FROM of the Dockerfile

\item[\code{AS}] AS of the Dockerfile
\end{ldescription}
\end{Arguments}
%
\begin{Value}
A Dockerfile Object.
\end{Value}
%
\begin{Note}\relax
After this Dockerfile is created, the package
should be built and be put in the same directory
as the dockerfile.
\end{Note}
%
\begin{Examples}
\begin{ExampleCode}
## Not run: 
my_dock <- dock_from_desc("DESCRIPTION")
my_dock
my_dock$CMD(r(library(dockerfiler)))
my_dock$add_after(
cmd = "RUN R -e 'remotes::install_cran(\"rlang\")'",
after = 3
)

## End(Not run)
\end{ExampleCode}
\end{Examples}
\inputencoding{utf8}
\HeaderA{r}{Turn an R call into an Unix call}{r}
%
\begin{Description}\relax
Turn an R call into an Unix call
\end{Description}
%
\begin{Usage}
\begin{verbatim}
r(code)
\end{verbatim}
\end{Usage}
%
\begin{Arguments}
\begin{ldescription}
\item[\code{code}] the function to call
\end{ldescription}
\end{Arguments}
%
\begin{Value}
an unix R call
\end{Value}
%
\begin{Examples}
\begin{ExampleCode}
r(print("yeay"))
r(install.packages("plumber", repo = "http://cran.irsn.fr/"))
\end{ExampleCode}
\end{Examples}
\printindex{}
\end{document}

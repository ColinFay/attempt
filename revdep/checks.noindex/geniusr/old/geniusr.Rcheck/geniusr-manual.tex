\nonstopmode{}
\documentclass[a4paper]{book}
\usepackage[times,inconsolata,hyper]{Rd}
\usepackage{makeidx}
\usepackage[utf8]{inputenc} % @SET ENCODING@
% \usepackage{graphicx} % @USE GRAPHICX@
\makeindex{}
\begin{document}
\chapter*{}
\begin{center}
{\textbf{\huge Package `geniusr'}}
\par\bigskip{\large \today}
\end{center}
\begin{description}
\raggedright{}
\inputencoding{utf8}
\item[Title]\AsIs{Tools for Working with the 'Genius' API}
\item[Version]\AsIs{1.1.0}
\item[Description]\AsIs{Provides tools to interact nicely with the 'Genius' API <https://docs.genius.com/>. Search hosted content, extract different types of associated metadata and retrieve lyrics with ease.}
\item[URL]\AsIs{}\url{https://github.com/ewenme/geniusr}\AsIs{}
\item[BugReports]\AsIs{}\url{https://github.com/ewenme/geniusr/issues}\AsIs{}
\item[Depends]\AsIs{R (>= 3.2.0)}
\item[License]\AsIs{MIT + file LICENSE}
\item[Encoding]\AsIs{UTF-8}
\item[LazyData]\AsIs{yes}
\item[RoxygenNote]\AsIs{6.1.1}
\item[Imports]\AsIs{httr, purrr, tibble, rvest, stringr, xml2, curl, attempt}
\item[NeedsCompilation]\AsIs{no}
\item[Author]\AsIs{Ewen Henderson [aut, cre]}
\item[Maintainer]\AsIs{Ewen Henderson }\email{ewenhenderson@gmail.com}\AsIs{}
\item[Repository]\AsIs{CRAN}
\item[Date/Publication]\AsIs{2019-01-20 21:50:02 UTC}
\end{description}
\Rdcontents{\R{} topics documented:}
\inputencoding{utf8}
\HeaderA{geniusr}{\code{geniusr} package}{geniusr}
\aliasA{geniusr-package}{geniusr}{geniusr.Rdash.package}
%
\begin{Description}\relax
An interface to Genius' Web API.
\end{Description}
%
\begin{Details}\relax
See the README on
\Rhref{https://github.com/ewenme/geniusr#readme}{GitHub}
\end{Details}
\inputencoding{utf8}
\HeaderA{genius\_token}{Get or set GENIUS\_API\_TOKEN value}{genius.Rul.token}
%
\begin{Description}\relax
The API wrapper functions in this package all rely on a Genius client access token
residing in the environment variable \code{GENIUS\_API\_TOKEN}. The
easiest way to accomplish this is to set it in the `\code{.Renviron}` file in your
home directory.
\end{Description}
%
\begin{Usage}
\begin{verbatim}
genius_token(force = FALSE)
\end{verbatim}
\end{Usage}
%
\begin{Arguments}
\begin{ldescription}
\item[\code{force}] force setting a new Genius API token for the current environment?
\end{ldescription}
\end{Arguments}
%
\begin{Value}
atomic character vector containing the Genius API token
\end{Value}
\inputencoding{utf8}
\HeaderA{get\_album\_meta}{Retrieve meta data for an album}{get.Rul.album.Rul.meta}
%
\begin{Description}\relax
The Genius API lets you search for an album's meta data, given an album ID.
\end{Description}
%
\begin{Usage}
\begin{verbatim}
get_album_meta(album_id, access_token = genius_token())
\end{verbatim}
\end{Usage}
%
\begin{Arguments}
\begin{ldescription}
\item[\code{album\_id}] An album ID (\code{album\_id} returned in \code{\LinkA{get\_song\_meta}{get.Rul.song.Rul.meta}})

\item[\code{access\_token}] Genius' client access token, defaults to \code{genius\_token}
\end{ldescription}
\end{Arguments}
%
\begin{Examples}
\begin{ExampleCode}
## Not run: 
get_album_meta(album_id = 337082)

## End(Not run)
\end{ExampleCode}
\end{Examples}
\inputencoding{utf8}
\HeaderA{get\_artist\_meta}{Retrieve meta data for an artist}{get.Rul.artist.Rul.meta}
%
\begin{Description}\relax
The Genius API lets you search for meta data for an artist, given an artist ID.
\end{Description}
%
\begin{Usage}
\begin{verbatim}
get_artist_meta(artist_id, access_token = genius_token())
\end{verbatim}
\end{Usage}
%
\begin{Arguments}
\begin{ldescription}
\item[\code{artist\_id}] An artist ID (\code{artist\_id} returned in \code{\LinkA{search\_artist}{search.Rul.artist}})

\item[\code{access\_token}] Genius' client access token, defaults to \code{genius\_token}
\end{ldescription}
\end{Arguments}
%
\begin{Examples}
\begin{ExampleCode}
## Not run: 
get_artist_meta(artist_id = 16751)

## End(Not run)
\end{ExampleCode}
\end{Examples}
\inputencoding{utf8}
\HeaderA{get\_artist\_songs}{Retrieve meta data for all of an artist's appearances on Genius}{get.Rul.artist.Rul.songs}
%
\begin{Description}\relax
Return meta data for all appearances (features optional) of an artist on Genius.
\end{Description}
%
\begin{Usage}
\begin{verbatim}
get_artist_songs(artist_id, include_features = FALSE,
  access_token = genius_token())
\end{verbatim}
\end{Usage}
%
\begin{Arguments}
\begin{ldescription}
\item[\code{artist\_id}] An artist ID (\code{artist\_id} returned in \code{\LinkA{search\_artist}{search.Rul.artist}})

\item[\code{include\_features}] Whether to return results where artist isn't the primary artist (logical, defaults to FALSE)

\item[\code{access\_token}] Genius' client access token, defaults to \code{genius\_token}
\end{ldescription}
\end{Arguments}
%
\begin{Examples}
\begin{ExampleCode}
## Not run: 
get_artist_songs(artist_id = 1421)

## End(Not run)
\end{ExampleCode}
\end{Examples}
\inputencoding{utf8}
\HeaderA{get\_song\_meta}{Retrieve meta data for a song}{get.Rul.song.Rul.meta}
%
\begin{Description}\relax
The Genius API lets you search for meta data for a song, given a song ID.
\end{Description}
%
\begin{Usage}
\begin{verbatim}
get_song_meta(song_id, access_token = genius_token())
\end{verbatim}
\end{Usage}
%
\begin{Arguments}
\begin{ldescription}
\item[\code{song\_id}] A song ID (\code{song\_id} returned in \code{\LinkA{search\_song}{search.Rul.song}})

\item[\code{access\_token}] Genius' client access token, defaults to \code{genius\_token}
\end{ldescription}
\end{Arguments}
%
\begin{Examples}
\begin{ExampleCode}
## Not run: 
get_song_meta(song_id = 3039923)

## End(Not run)
\end{ExampleCode}
\end{Examples}
\inputencoding{utf8}
\HeaderA{scrape\_lyrics\_id}{Retrieve lyrics associated with a Genius song ID}{scrape.Rul.lyrics.Rul.id}
%
\begin{Description}\relax
Scrape lyrics from Genius' lyric pages using an associated song ID.
\end{Description}
%
\begin{Usage}
\begin{verbatim}
scrape_lyrics_id(song_id, access_token = genius_token())
\end{verbatim}
\end{Usage}
%
\begin{Arguments}
\begin{ldescription}
\item[\code{song\_id}] song ID (like in \code{song\_id} returned by \code{\LinkA{search\_song}{search.Rul.song}})

\item[\code{access\_token}] Genius' client access token, defaults to \code{genius\_token}
\end{ldescription}
\end{Arguments}
%
\begin{Examples}
\begin{ExampleCode}
## Not run: 
scrape_lyrics_id(song_id = 3214267)

## End(Not run)
\end{ExampleCode}
\end{Examples}
\inputencoding{utf8}
\HeaderA{scrape\_lyrics\_url}{Retrieve lyrics associated with a Genius lyrics page URL}{scrape.Rul.lyrics.Rul.url}
%
\begin{Description}\relax
Scrape lyrics from a Genius' lyric page using it's associated URL. Best used with \code{\LinkA{scrape\_tracklist}{scrape.Rul.tracklist}}, when song IDs aren't returned - otherwise, \code{\LinkA{scrape\_lyrics\_id}{scrape.Rul.lyrics.Rul.id}} is recommended.
\end{Description}
%
\begin{Usage}
\begin{verbatim}
scrape_lyrics_url(song_lyrics_url, access_token = genius_token())
\end{verbatim}
\end{Usage}
%
\begin{Arguments}
\begin{ldescription}
\item[\code{song\_lyrics\_url}] song lyrics url (like in \code{song\_lyrics\_url} returned by \code{\LinkA{get\_song\_meta}{get.Rul.song.Rul.meta}})

\item[\code{access\_token}] Genius' client access token, defaults to \code{genius\_token}
\end{ldescription}
\end{Arguments}
%
\begin{Examples}
\begin{ExampleCode}
## Not run: 
scrape_lyrics_url(song_lyrics_url = "https://genius.com/Kendrick-lamar-dna-lyrics")

## End(Not run)
\end{ExampleCode}
\end{Examples}
\inputencoding{utf8}
\HeaderA{scrape\_tracklist}{Retrieve an album's tracklisting}{scrape.Rul.tracklist}
%
\begin{Description}\relax
Scrape an album's tracklisting, and song meta data, given an album ID.
\end{Description}
%
\begin{Usage}
\begin{verbatim}
scrape_tracklist(album_id, access_token = genius_token())
\end{verbatim}
\end{Usage}
%
\begin{Arguments}
\begin{ldescription}
\item[\code{album\_id}] An album ID (\code{album\_id} returned in \code{\LinkA{get\_song\_meta}{get.Rul.song.Rul.meta}})

\item[\code{access\_token}] Genius' client access token, defaults to \code{genius\_token}
\end{ldescription}
\end{Arguments}
%
\begin{Examples}
\begin{ExampleCode}
## Not run: 
scrape_tracklist(album_id = 337082)

## End(Not run)
\end{ExampleCode}
\end{Examples}
\inputencoding{utf8}
\HeaderA{search\_artist}{Retrieve artist identifiers for associated search terms}{search.Rul.artist}
%
\begin{Description}\relax
The Genius API lets you search hosted content (all songs). Use \code{search\_artist} to
return \code{artist\_id}, \code{artist\_name} and \code{artist\_url} for all unique artist matches to a search.
\end{Description}
%
\begin{Usage}
\begin{verbatim}
search_artist(search_term, n_results = 10,
  access_token = genius_token())
\end{verbatim}
\end{Usage}
%
\begin{Arguments}
\begin{ldescription}
\item[\code{search\_term}] A character string to search for artist matches

\item[\code{n\_results}] Maximum no. of search results to return (this is the number of hosted content search results, unique artist matches will be smaller)

\item[\code{access\_token}] Genius' client access token, defaults to \code{genius\_token}
\end{ldescription}
\end{Arguments}
%
\begin{Examples}
\begin{ExampleCode}
## Not run: 
search_artist(search_term = "Lil")

## End(Not run)
\end{ExampleCode}
\end{Examples}
\inputencoding{utf8}
\HeaderA{search\_song}{Retrieve song identifiers for associated search terms}{search.Rul.song}
%
\begin{Description}\relax
The Genius API lets you search hosted content (all songs). Use \code{search\_song} to
return \code{song\_id}, \code{song\_name}, \code{lyrics\_url} and
\code{artist\_id} for all unique song matches to a search.
\end{Description}
%
\begin{Usage}
\begin{verbatim}
search_song(search_term, n_results = 10, access_token = genius_token())
\end{verbatim}
\end{Usage}
%
\begin{Arguments}
\begin{ldescription}
\item[\code{search\_term}] A character string to search for song matches

\item[\code{n\_results}] Maximum no. of search results to return

\item[\code{access\_token}] Genius' client access token, defaults to \code{genius\_token}
\end{ldescription}
\end{Arguments}
%
\begin{Examples}
\begin{ExampleCode}
## Not run: 
search_song(search_term = "Gucci", n_results=50)

## End(Not run)
\end{ExampleCode}
\end{Examples}
\printindex{}
\end{document}

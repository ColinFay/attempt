\nonstopmode{}
\documentclass[a4paper]{book}
\usepackage[times,inconsolata,hyper]{Rd}
\usepackage{makeidx}
\usepackage[utf8,latin1]{inputenc}
% \usepackage{graphicx} % @USE GRAPHICX@
\makeindex{}
\begin{document}
\chapter*{}
\begin{center}
{\textbf{\huge Package `languagelayeR'}}
\par\bigskip{\large \today}
\end{center}
\begin{description}
\raggedright{}
\item[Type]\AsIs{Package}
\item[Title]\AsIs{Access the 'languagelayer' API}
\item[Version]\AsIs{1.2.4}
\item[Maintainer]\AsIs{Colin FAY }\email{contact@colinfay.me}\AsIs{}
\item[Description]\AsIs{Improve your text analysis with languagelayer
<https://languagelayer.com>, a powerful language detection
API.}
\item[URL]\AsIs{}\url{https://github.com/ColinFay/languagelayer}\AsIs{}
\item[BugReports]\AsIs{}\url{https://github.com/ColinFay/languagelayer/issues}\AsIs{}
\item[License]\AsIs{MIT + file LICENSE}
\item[LazyData]\AsIs{TRUE}
\item[Imports]\AsIs{httr, jsonlite, attempt, curl, utils}
\item[RoxygenNote]\AsIs{6.1.0}
\item[Suggests]\AsIs{testthat, knitr, rmarkdown, covr}
\item[ByteCompile]\AsIs{true}
\item[VignetteBuilder]\AsIs{knitr}
\item[NeedsCompilation]\AsIs{no}
\item[Author]\AsIs{Colin FAY [aut, cre]}
\item[Repository]\AsIs{CRAN}
\item[Date/Publication]\AsIs{2019-02-12 10:03:19 UTC}
\end{description}
\Rdcontents{\R{} topics documented:}
\inputencoding{utf8}
\HeaderA{get\_lang}{Get language}{get.Rul.lang}
%
\begin{Description}\relax
Detect language from a character string.
\end{Description}
%
\begin{Usage}
\begin{verbatim}
get_lang(query, api_key = NULL)
\end{verbatim}
\end{Usage}
%
\begin{Arguments}
\begin{ldescription}
\item[\code{query}] the character string you want to detect the language from.

\item[\code{api\_key}] your API key.
\end{ldescription}
\end{Arguments}
%
\begin{Details}\relax
Takes a character string, returns a data.frame with the available values.
\end{Details}
%
\begin{Value}
Returns a data.frame with the detected languages, in descending order of probability. Values are : language\_code, language\_name, probability (length of the provided query text and how well it is identified as a language), percentage (confidence margin between multiple matches), and reliable\_result (whether or not the API is completely confident about the main match).
\end{Value}
%
\begin{Note}\relax
Before running a function of this package for the first time, you need to get your API key.
\end{Note}
%
\begin{Examples}
\begin{ExampleCode}
## Not run: 
get_lang(query = "I really really love R and that's a good thing, right?", api_key = "apikey")

## End(Not run)
\end{ExampleCode}
\end{Examples}
\inputencoding{utf8}
\HeaderA{get\_supported\_lang}{Get supported languages}{get.Rul.supported.Rul.lang}
%
\begin{Description}\relax
Get all current available languages on the languagelayer API.
\end{Description}
%
\begin{Usage}
\begin{verbatim}
get_supported_lang(api_key = NULL)
\end{verbatim}
\end{Usage}
%
\begin{Arguments}
\begin{ldescription}
\item[\code{api\_key}] Your API key.
\end{ldescription}
\end{Arguments}
%
\begin{Details}\relax
Returns a data.frame with the available languages.
\end{Details}
%
\begin{Value}
Returns a data.frame with language\_code and language\_name.
\end{Value}
%
\begin{Note}\relax
Before running a function of this package for the first time, you need to get your API key.
\end{Note}
%
\begin{Examples}
\begin{ExampleCode}
## Not run: 
get_supported_lang(api_key = "yourapikey")

## End(Not run)
\end{ExampleCode}
\end{Examples}
\inputencoding{utf8}
\HeaderA{languagelayeR}{languagelayeR}{languagelayeR}
%
\begin{Description}\relax
Access the 'languagelayer' API.
\end{Description}
%
\begin{Details}\relax
For more info \code{browseVignettes("languagelayeR")}.
\end{Details}
\printindex{}
\end{document}

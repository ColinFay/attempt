\nonstopmode{}
\documentclass[a4paper]{book}
\usepackage[times,inconsolata,hyper]{Rd}
\usepackage{makeidx}
\usepackage[utf8]{inputenc} % @SET ENCODING@
% \usepackage{graphicx} % @USE GRAPHICX@
\makeindex{}
\begin{document}
\chapter*{}
\begin{center}
{\textbf{\huge Package `proustr'}}
\par\bigskip{\large \today}
\end{center}
\begin{description}
\raggedright{}
\inputencoding{utf8}
\item[Title]\AsIs{Tools for Natural Language Processing in French}
\item[Version]\AsIs{0.4.0}
\item[Date]\AsIs{2019-02-05}
\item[Description]\AsIs{Tools for Natural Language Processing in French and texts from Marcel Proust's collection
``A La Recherche Du Temps Perdu''. The novels contained in this collection are
``Du cote de chez Swann '', ``A l'ombre des jeunes filles en fleurs'',``Le Cote de Guermantes'',
``Sodome et Gomorrhe I et II'', ``La Prisonniere'', ``Albertine disparue'', and ``Le Temps retrouve''.}
\item[URL]\AsIs{}\url{https://github.com/ColinFay/proustr}\AsIs{}
\item[BugReports]\AsIs{}\url{https://github.com/ColinFay/proustr/issues}\AsIs{}
\item[Depends]\AsIs{R (>= 2.10)}
\item[License]\AsIs{MIT + file LICENSE}
\item[Imports]\AsIs{stringr, rlang, tidyr, tokenizers, SnowballC, attempt}
\item[LazyData]\AsIs{true}
\item[RoxygenNote]\AsIs{6.1.0}
\item[Encoding]\AsIs{UTF-8}
\item[Suggests]\AsIs{testthat, knitr, rmarkdown, covr, dplyr}
\item[VignetteBuilder]\AsIs{knitr}
\item[NeedsCompilation]\AsIs{no}
\item[Author]\AsIs{Colin Fay [aut, cre] (<https://orcid.org/0000-0001-7343-1846>)}
\item[Maintainer]\AsIs{Colin Fay }\email{contact@colinfay.me}\AsIs{}
\item[Repository]\AsIs{CRAN}
\item[Date/Publication]\AsIs{2019-02-05 14:50:02 UTC}
\end{description}
\Rdcontents{\R{} topics documented:}
\inputencoding{utf8}
\HeaderA{albertinedisparue}{Marcel Proust's novel "Albertine disparue"}{albertinedisparue}
\keyword{datasets}{albertinedisparue}
%
\begin{Description}\relax
A dataset containing Marcel Proust's "Albertine disparue". 
This text has been downloaded from WikiSource.
\end{Description}
%
\begin{Usage}
\begin{verbatim}
albertinedisparue
\end{verbatim}
\end{Usage}
%
\begin{Format}
A tibble with text, book, volume, and year
\end{Format}
%
\begin{Source}\relax
<https://fr.wikisource.org/wiki/Albertine\_disparue>
\end{Source}
\inputencoding{utf8}
\HeaderA{alombredesjeunesfillesenfleurs}{Marcel Proust's novel "À l’ombre des jeunes filles en fleurs"}{alombredesjeunesfillesenfleurs}
\keyword{datasets}{alombredesjeunesfillesenfleurs}
%
\begin{Description}\relax
A dataset containing Marcel Proust's "À l’ombre des jeunes filles en fleurs". 
This text has been downloaded from WikiSource.
\end{Description}
%
\begin{Usage}
\begin{verbatim}
alombredesjeunesfillesenfleurs
\end{verbatim}
\end{Usage}
%
\begin{Format}
A tibble with text, book, volume, and year
\end{Format}
%
\begin{Source}\relax
<https://fr.wikisource.org/wiki/
\end{Source}
\inputencoding{utf8}
\HeaderA{ducotedechezswann}{Marcel Proust's novel "Du côté de chez Swann"}{ducotedechezswann}
\keyword{datasets}{ducotedechezswann}
%
\begin{Description}\relax
A dataset containing Marcel Proust's "Du côté de chez Swann". 
This text has been downloaded from WikiSource.
\end{Description}
%
\begin{Usage}
\begin{verbatim}
ducotedechezswann
\end{verbatim}
\end{Usage}
%
\begin{Format}
A tibble with text, book, volume, and year
\end{Format}
%
\begin{Source}\relax
<https://fr.wikisource.org/wiki/Du\_c
\end{Source}
\inputencoding{utf8}
\HeaderA{laprisonniere}{Marcel Proust's novel "La Prisonnière"}{laprisonniere}
\keyword{datasets}{laprisonniere}
%
\begin{Description}\relax
A dataset containing Marcel Proust's "La prisonnière". 
This text has been downloaded from WikiSource.
\end{Description}
%
\begin{Usage}
\begin{verbatim}
laprisonniere
\end{verbatim}
\end{Usage}
%
\begin{Format}
A tibble with text, book, volume, and year
\end{Format}
%
\begin{Source}\relax
<https://fr.wikisource.org/wiki/La\_Prisonni
\end{Source}
\inputencoding{utf8}
\HeaderA{lecotedeguermantes}{Marcel Proust's novel "Le côté de Guermantes"}{lecotedeguermantes}
\keyword{datasets}{lecotedeguermantes}
%
\begin{Description}\relax
A dataset containing Marcel Proust's "À l’ombre des jeunes filles en fleurs". 
This text has been downloaded from WikiSource.
\end{Description}
%
\begin{Usage}
\begin{verbatim}
lecotedeguermantes
\end{verbatim}
\end{Usage}
%
\begin{Format}
A tibble with text, book, volume, and year
\end{Format}
%
\begin{Source}\relax
<https://fr.wikisource.org/wiki/Le\_C
\end{Source}
\inputencoding{utf8}
\HeaderA{letempretrouve}{Marcel Proust's novel "Le temps retrouvé"}{letempretrouve}
\keyword{datasets}{letempretrouve}
%
\begin{Description}\relax
A dataset containing Marcel Proust's "Le temps retrouvé". 
This text has been downloaded from WikiSource.
\end{Description}
%
\begin{Usage}
\begin{verbatim}
letempretrouve
\end{verbatim}
\end{Usage}
%
\begin{Format}
A tibble with text, book, volume, and year.
\end{Format}
%
\begin{Source}\relax
<https://fr.wikisource.org/wiki/Le\_Temps\_retrouv
\end{Source}
\inputencoding{utf8}
\HeaderA{proust\_books}{Tidy data frame of Marcel Proust's 7 novels from La Recherche}{proust.Rul.books}
%
\begin{Description}\relax
Returns a tidy tibble of Marcel Proust's 7 novels from À la recherche du temps
perdu. The tibble contains four columns: text, book, volume and year.
\end{Description}
%
\begin{Usage}
\begin{verbatim}
proust_books()
\end{verbatim}
\end{Usage}
%
\begin{Value}
A tibble with four columns: \code{text}, \code{book}, \code{volume} and \code{year}.
\end{Value}
%
\begin{Examples}
\begin{ExampleCode}

#Create the tibble 
proust <- proust_books()
 

\end{ExampleCode}
\end{Examples}
\inputencoding{utf8}
\HeaderA{proust\_char}{Characters from "À la recherche du temps perdu"}{proust.Rul.char}
\keyword{datasets}{proust\_char}
%
\begin{Description}\relax
A dataset containing Marcel Proust's characters from "À la recherche du temps perdu"  and 
their frequency in each book. 
This dataset has been downloaded from proust-personnages.
\end{Description}
%
\begin{Usage}
\begin{verbatim}
proust_char
\end{verbatim}
\end{Usage}
%
\begin{Format}
A tibble with their name
\end{Format}
%
\begin{Source}\relax
\url{http://proust-personnages.fr/?page_id=10254}
\end{Source}
\inputencoding{utf8}
\HeaderA{proust\_characters}{Characters from Proust Books}{proust.Rul.characters}
%
\begin{Description}\relax
Returns a tidy data frame of Marcel Proust's characters.
\end{Description}
%
\begin{Usage}
\begin{verbatim}
proust_characters()
\end{verbatim}
\end{Usage}
%
\begin{Value}
A tibble
\end{Value}
%
\begin{Source}\relax
\url{http://proust-personnages.fr/}
\end{Source}
%
\begin{Examples}
\begin{ExampleCode}

#Creates the tibble 
proust <- proust_characters()
 

\end{ExampleCode}
\end{Examples}
\inputencoding{utf8}
\HeaderA{proust\_random}{Create a Random Proust extract}{proust.Rul.random}
%
\begin{Description}\relax
Create your own flavor of Proust with this random extractor.
\end{Description}
%
\begin{Usage}
\begin{verbatim}
proust_random(count = 1, collapse = TRUE)
\end{verbatim}
\end{Usage}
%
\begin{Arguments}
\begin{ldescription}
\item[\code{count}] the number of line you want to randomly extract and paste.

\item[\code{collapse}] if FALSE, the output will be a tibble. Default is TRUE, a character vector.
\end{ldescription}
\end{Arguments}
%
\begin{Value}
a character vector
\end{Value}
%
\begin{Examples}
\begin{ExampleCode}
proust_random(4)
\end{ExampleCode}
\end{Examples}
\inputencoding{utf8}
\HeaderA{proust\_sentiments}{Old sentiment lexicon This function has been deprecated, and will be in next proustr version. See the rfeel package now: http://github.com/ColinFay/rfeel}{proust.Rul.sentiments}
%
\begin{Description}\relax
Old sentiment lexicon
This function has been deprecated, and will be in next proustr version.
See the rfeel package now: http://github.com/ColinFay/rfeel
\end{Description}
%
\begin{Usage}
\begin{verbatim}
proust_sentiments(type = c("polarity", "score"))
\end{verbatim}
\end{Usage}
%
\begin{Arguments}
\begin{ldescription}
\item[\code{type}] For backward compatibility
\end{ldescription}
\end{Arguments}
%
\begin{Value}
a tibble
\end{Value}
\inputencoding{utf8}
\HeaderA{proust\_stopwords}{Stop Words}{proust.Rul.stopwords}
%
\begin{Description}\relax
Stop words concatenated from various web sources.
\end{Description}
%
\begin{Usage}
\begin{verbatim}
proust_stopwords()
\end{verbatim}
\end{Usage}
%
\begin{Value}
a tibble with stopwords
\end{Value}
%
\begin{Source}\relax
\url{https://raw.githubusercontent.com/stopwords-iso/stopwords-fr/master/stopwords-fr.txt}
\end{Source}
%
\begin{Examples}
\begin{ExampleCode}
proust_stopwords()
\end{ExampleCode}
\end{Examples}
\inputencoding{utf8}
\HeaderA{pr\_detect\_days}{Detect french days}{pr.Rul.detect.Rul.days}
%
\begin{Description}\relax
Detect the name of the days (in French)
\end{Description}
%
\begin{Usage}
\begin{verbatim}
pr_detect_days(df, col)
\end{verbatim}
\end{Usage}
%
\begin{Arguments}
\begin{ldescription}
\item[\code{df}] a dataframe

\item[\code{col}] the column containing the text
\end{ldescription}
\end{Arguments}
%
\begin{Value}
a tibble with the number of days detected by the algo
\end{Value}
%
\begin{Examples}
\begin{ExampleCode}
a <- data.frame(jours = c("C'est lundi 1er mars et mardi 2", 
"Et mercredi 3", "Il est revenu jeudi."))
pr_detect_days(a, jours)
\end{ExampleCode}
\end{Examples}
\inputencoding{utf8}
\HeaderA{pr\_detect\_months}{Detect french months}{pr.Rul.detect.Rul.months}
%
\begin{Description}\relax
Detect the name of the months (in French)
\end{Description}
%
\begin{Usage}
\begin{verbatim}
pr_detect_months(df, col)
\end{verbatim}
\end{Usage}
%
\begin{Arguments}
\begin{ldescription}
\item[\code{df}] a dataframe

\item[\code{col}] the column containing the text
\end{ldescription}
\end{Arguments}
%
\begin{Value}
a tibble with the number of days detected by the algo
\end{Value}
%
\begin{Examples}
\begin{ExampleCode}
a <- data.frame(month = c("C'est lundi 1er mars et mardi 2", 
"Et mercredi 3", "Il est revenu en juin."))
pr_detect_months(a, month)
\end{ExampleCode}
\end{Examples}
\inputencoding{utf8}
\HeaderA{pr\_detect\_pro}{Detect French pronoums}{pr.Rul.detect.Rul.pro}
%
\begin{Description}\relax
Detect the pronouns from a text (in French)
\end{Description}
%
\begin{Usage}
\begin{verbatim}
pr_detect_pro(df, col, verbose = FALSE)
\end{verbatim}
\end{Usage}
%
\begin{Arguments}
\begin{ldescription}
\item[\code{df}] a dataframe

\item[\code{col}] the column containing the text

\item[\code{verbose}] wether or not to return the list of pronouns. Defaults is FALSE
\end{ldescription}
\end{Arguments}
%
\begin{Details}\relax
The shortcuts in the pronoun col stand for: 

pps: first person singular (première personne du singulier)

dps: second person singular (deuxième personne du singulier)

tps: third person singular (troisième personne du singulier)

ppp: first person plural (première personne du pluriel)

dpp: second person singular (deuxième personne du pluriel)

tpp: third person singular (troisième personne du pluriel)
\end{Details}
%
\begin{Value}
a tibble with the detected pronouns
\end{Value}
%
\begin{Examples}
\begin{ExampleCode}
library(proustr)
a <- proust_books()[1,] 
pr_detect_pro(a, text, verbose = TRUE)
pr_detect_pro(a, text)
\end{ExampleCode}
\end{Examples}
\inputencoding{utf8}
\HeaderA{pr\_keep\_only\_alnum}{Remove non alnum elements}{pr.Rul.keep.Rul.only.Rul.alnum}
%
\begin{Description}\relax
Remove non alnum elements
\end{Description}
%
\begin{Usage}
\begin{verbatim}
pr_keep_only_alnum(text, replacement = " ")
\end{verbatim}
\end{Usage}
%
\begin{Arguments}
\begin{ldescription}
\item[\code{text}] a vector

\item[\code{replacement}] what to replace the non alnum with. Defaut is " ".
\end{ldescription}
\end{Arguments}
%
\begin{Value}
a vector
\end{Value}
%
\begin{Examples}
\begin{ExampleCode}
pr_keep_only_alnum("neuilly-en-thelle")
\end{ExampleCode}
\end{Examples}
\inputencoding{utf8}
\HeaderA{pr\_normalize\_punc}{Normalize punctuation}{pr.Rul.normalize.Rul.punc}
%
\begin{Description}\relax
Normalize a text written with usual french punctuation
\end{Description}
%
\begin{Usage}
\begin{verbatim}
pr_normalize_punc(df, col)
\end{verbatim}
\end{Usage}
%
\begin{Arguments}
\begin{ldescription}
\item[\code{df}] a dataframe

\item[\code{col}] the column to normalize
\end{ldescription}
\end{Arguments}
%
\begin{Value}
a tibble with normalized text
\end{Value}
%
\begin{Examples}
\begin{ExampleCode}
a <- proustr::albertinedisparue[1:20,]
pr_normalize_punc(albertinedisparue, text)
\end{ExampleCode}
\end{Examples}
\inputencoding{utf8}
\HeaderA{pr\_stem\_sentences}{Stem a dataframe containing a column with sentences}{pr.Rul.stem.Rul.sentences}
%
\begin{Description}\relax
Implementation of the SnowballC stemmer. Note that punctuation and capital letters 
are removed when processing.
\end{Description}
%
\begin{Usage}
\begin{verbatim}
pr_stem_sentences(df, col, language = "french")
\end{verbatim}
\end{Usage}
%
\begin{Arguments}
\begin{ldescription}
\item[\code{df}] the data.frame containing the text

\item[\code{col}] the column with the text

\item[\code{language}] the language of the text. Defaut is french. See SnowballC::getStemLanguages() function for a list of supported languages.
\end{ldescription}
\end{Arguments}
%
\begin{Value}
a tibble
\end{Value}
%
\begin{Examples}
\begin{ExampleCode}
a <- proustr::laprisonniere[1:10,]
pr_stem_sentences(a, text)

\end{ExampleCode}
\end{Examples}
\inputencoding{utf8}
\HeaderA{pr\_stem\_words}{Stem a dataframe containing a column with words}{pr.Rul.stem.Rul.words}
%
\begin{Description}\relax
Implementation of the SnowballC stemmer. Note that punctuation and capitals letters 
are also removed.
\end{Description}
%
\begin{Usage}
\begin{verbatim}
pr_stem_words(df, col, language = "french")
\end{verbatim}
\end{Usage}
%
\begin{Arguments}
\begin{ldescription}
\item[\code{df}] the data.frame containing the sentences

\item[\code{col}] the column with the sentences

\item[\code{language}] the language of the words Defaut is french. See SnowballC::getStemLanguages() function for a list of supported languages.
\end{ldescription}
\end{Arguments}
%
\begin{Value}
a tibble
\end{Value}
%
\begin{Examples}
\begin{ExampleCode}
a <- data.frame(words = c("matin", "heure", "fatigué","sonné","lois", "tests","fusionner"))
pr_stem_words(a, words)

\end{ExampleCode}
\end{Examples}
\inputencoding{utf8}
\HeaderA{pr\_unacent}{Remove accents}{pr.Rul.unacent}
%
\begin{Description}\relax
Remove accents from a character vector
\end{Description}
%
\begin{Usage}
\begin{verbatim}
pr_unacent(text)
\end{verbatim}
\end{Usage}
%
\begin{Arguments}
\begin{ldescription}
\item[\code{text}] a vector
\end{ldescription}
\end{Arguments}
%
\begin{Value}
a vector
\end{Value}
%
\begin{Examples}
\begin{ExampleCode}
pr_unacent("du chêne")
\end{ExampleCode}
\end{Examples}
\inputencoding{utf8}
\HeaderA{sodomeetgomorrhe}{Marcel Proust's novel "Sodome et Gomorrhe"}{sodomeetgomorrhe}
\keyword{datasets}{sodomeetgomorrhe}
%
\begin{Description}\relax
A dataset containing Marcel Proust's "Sodom et Gomorrhe". 
This text has been downloaded from WikiSource.
\end{Description}
%
\begin{Usage}
\begin{verbatim}
sodomeetgomorrhe
\end{verbatim}
\end{Usage}
%
\begin{Format}
A tibble with text, book, volume, and year
\end{Format}
%
\begin{Source}\relax
<https://fr.wikisource.org/wiki/Sodome\_et\_Gomorrhe>
\end{Source}
\inputencoding{utf8}
\HeaderA{stop\_words}{Stopwords}{stop.Rul.words}
\keyword{datasets}{stop\_words}
%
\begin{Description}\relax
ISO stopwords
\end{Description}
%
\begin{Usage}
\begin{verbatim}
stop_words
\end{verbatim}
\end{Usage}
%
\begin{Format}
A tibble
\end{Format}
%
\begin{Source}\relax
\url{https://raw.githubusercontent.com/stopwords-iso/stopwords-iso/master/stopwords-iso.json}
\end{Source}
\printindex{}
\end{document}
